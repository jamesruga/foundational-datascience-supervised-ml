\documentclass[11pt]{article}

    \usepackage[breakable]{tcolorbox}
    \usepackage{parskip} % Stop auto-indenting (to mimic markdown behaviour)
    

    % Basic figure setup, for now with no caption control since it's done
    % automatically by Pandoc (which extracts ![](path) syntax from Markdown).
    \usepackage{graphicx}
    % Maintain compatibility with old templates. Remove in nbconvert 6.0
    \let\Oldincludegraphics\includegraphics
    % Ensure that by default, figures have no caption (until we provide a
    % proper Figure object with a Caption API and a way to capture that
    % in the conversion process - todo).
    \usepackage{caption}
    \DeclareCaptionFormat{nocaption}{}
    \captionsetup{format=nocaption,aboveskip=0pt,belowskip=0pt}

    \usepackage{float}
    \floatplacement{figure}{H} % forces figures to be placed at the correct location
    \usepackage{xcolor} % Allow colors to be defined
    \usepackage{enumerate} % Needed for markdown enumerations to work
    \usepackage{geometry} % Used to adjust the document margins
    \usepackage{amsmath} % Equations
    \usepackage{amssymb} % Equations
    \usepackage{textcomp} % defines textquotesingle
    % Hack from http://tex.stackexchange.com/a/47451/13684:
    \AtBeginDocument{%
        \def\PYZsq{\textquotesingle}% Upright quotes in Pygmentized code
    }
    \usepackage{upquote} % Upright quotes for verbatim code
    \usepackage{eurosym} % defines \euro

    \usepackage{iftex}
    \ifPDFTeX
        \usepackage[T1]{fontenc}
        \IfFileExists{alphabeta.sty}{
              \usepackage{alphabeta}
          }{
              \usepackage[mathletters]{ucs}
              \usepackage[utf8x]{inputenc}
          }
    \else
        \usepackage{fontspec}
        \usepackage{unicode-math}
    \fi

    \usepackage{fancyvrb} % verbatim replacement that allows latex
    \usepackage{grffile} % extends the file name processing of package graphics
                         % to support a larger range
    \makeatletter % fix for old versions of grffile with XeLaTeX
    \@ifpackagelater{grffile}{2019/11/01}
    {
      % Do nothing on new versions
    }
    {
      \def\Gread@@xetex#1{%
        \IfFileExists{"\Gin@base".bb}%
        {\Gread@eps{\Gin@base.bb}}%
        {\Gread@@xetex@aux#1}%
      }
    }
    \makeatother
    \usepackage[Export]{adjustbox} % Used to constrain images to a maximum size
    \adjustboxset{max size={0.9\linewidth}{0.9\paperheight}}

    % The hyperref package gives us a pdf with properly built
    % internal navigation ('pdf bookmarks' for the table of contents,
    % internal cross-reference links, web links for URLs, etc.)
    \usepackage{hyperref}
    % The default LaTeX title has an obnoxious amount of whitespace. By default,
    % titling removes some of it. It also provides customization options.
    \usepackage{titling}
    \usepackage{longtable} % longtable support required by pandoc >1.10
    \usepackage{booktabs}  % table support for pandoc > 1.12.2
    \usepackage{array}     % table support for pandoc >= 2.11.3
    \usepackage{calc}      % table minipage width calculation for pandoc >= 2.11.1
    \usepackage[inline]{enumitem} % IRkernel/repr support (it uses the enumerate* environment)
    \usepackage[normalem]{ulem} % ulem is needed to support strikethroughs (\sout)
                                % normalem makes italics be italics, not underlines
    \usepackage{mathrsfs}
    

    
    % Colors for the hyperref package
    \definecolor{urlcolor}{rgb}{0,.145,.698}
    \definecolor{linkcolor}{rgb}{.71,0.21,0.01}
    \definecolor{citecolor}{rgb}{.12,.54,.11}

    % ANSI colors
    \definecolor{ansi-black}{HTML}{3E424D}
    \definecolor{ansi-black-intense}{HTML}{282C36}
    \definecolor{ansi-red}{HTML}{E75C58}
    \definecolor{ansi-red-intense}{HTML}{B22B31}
    \definecolor{ansi-green}{HTML}{00A250}
    \definecolor{ansi-green-intense}{HTML}{007427}
    \definecolor{ansi-yellow}{HTML}{DDB62B}
    \definecolor{ansi-yellow-intense}{HTML}{B27D12}
    \definecolor{ansi-blue}{HTML}{208FFB}
    \definecolor{ansi-blue-intense}{HTML}{0065CA}
    \definecolor{ansi-magenta}{HTML}{D160C4}
    \definecolor{ansi-magenta-intense}{HTML}{A03196}
    \definecolor{ansi-cyan}{HTML}{60C6C8}
    \definecolor{ansi-cyan-intense}{HTML}{258F8F}
    \definecolor{ansi-white}{HTML}{C5C1B4}
    \definecolor{ansi-white-intense}{HTML}{A1A6B2}
    \definecolor{ansi-default-inverse-fg}{HTML}{FFFFFF}
    \definecolor{ansi-default-inverse-bg}{HTML}{000000}

    % common color for the border for error outputs.
    \definecolor{outerrorbackground}{HTML}{FFDFDF}

    % commands and environments needed by pandoc snippets
    % extracted from the output of `pandoc -s`
    \providecommand{\tightlist}{%
      \setlength{\itemsep}{0pt}\setlength{\parskip}{0pt}}
    \DefineVerbatimEnvironment{Highlighting}{Verbatim}{commandchars=\\\{\}}
    % Add ',fontsize=\small' for more characters per line
    \newenvironment{Shaded}{}{}
    \newcommand{\KeywordTok}[1]{\textcolor[rgb]{0.00,0.44,0.13}{\textbf{{#1}}}}
    \newcommand{\DataTypeTok}[1]{\textcolor[rgb]{0.56,0.13,0.00}{{#1}}}
    \newcommand{\DecValTok}[1]{\textcolor[rgb]{0.25,0.63,0.44}{{#1}}}
    \newcommand{\BaseNTok}[1]{\textcolor[rgb]{0.25,0.63,0.44}{{#1}}}
    \newcommand{\FloatTok}[1]{\textcolor[rgb]{0.25,0.63,0.44}{{#1}}}
    \newcommand{\CharTok}[1]{\textcolor[rgb]{0.25,0.44,0.63}{{#1}}}
    \newcommand{\StringTok}[1]{\textcolor[rgb]{0.25,0.44,0.63}{{#1}}}
    \newcommand{\CommentTok}[1]{\textcolor[rgb]{0.38,0.63,0.69}{\textit{{#1}}}}
    \newcommand{\OtherTok}[1]{\textcolor[rgb]{0.00,0.44,0.13}{{#1}}}
    \newcommand{\AlertTok}[1]{\textcolor[rgb]{1.00,0.00,0.00}{\textbf{{#1}}}}
    \newcommand{\FunctionTok}[1]{\textcolor[rgb]{0.02,0.16,0.49}{{#1}}}
    \newcommand{\RegionMarkerTok}[1]{{#1}}
    \newcommand{\ErrorTok}[1]{\textcolor[rgb]{1.00,0.00,0.00}{\textbf{{#1}}}}
    \newcommand{\NormalTok}[1]{{#1}}

    % Additional commands for more recent versions of Pandoc
    \newcommand{\ConstantTok}[1]{\textcolor[rgb]{0.53,0.00,0.00}{{#1}}}
    \newcommand{\SpecialCharTok}[1]{\textcolor[rgb]{0.25,0.44,0.63}{{#1}}}
    \newcommand{\VerbatimStringTok}[1]{\textcolor[rgb]{0.25,0.44,0.63}{{#1}}}
    \newcommand{\SpecialStringTok}[1]{\textcolor[rgb]{0.73,0.40,0.53}{{#1}}}
    \newcommand{\ImportTok}[1]{{#1}}
    \newcommand{\DocumentationTok}[1]{\textcolor[rgb]{0.73,0.13,0.13}{\textit{{#1}}}}
    \newcommand{\AnnotationTok}[1]{\textcolor[rgb]{0.38,0.63,0.69}{\textbf{\textit{{#1}}}}}
    \newcommand{\CommentVarTok}[1]{\textcolor[rgb]{0.38,0.63,0.69}{\textbf{\textit{{#1}}}}}
    \newcommand{\VariableTok}[1]{\textcolor[rgb]{0.10,0.09,0.49}{{#1}}}
    \newcommand{\ControlFlowTok}[1]{\textcolor[rgb]{0.00,0.44,0.13}{\textbf{{#1}}}}
    \newcommand{\OperatorTok}[1]{\textcolor[rgb]{0.40,0.40,0.40}{{#1}}}
    \newcommand{\BuiltInTok}[1]{{#1}}
    \newcommand{\ExtensionTok}[1]{{#1}}
    \newcommand{\PreprocessorTok}[1]{\textcolor[rgb]{0.74,0.48,0.00}{{#1}}}
    \newcommand{\AttributeTok}[1]{\textcolor[rgb]{0.49,0.56,0.16}{{#1}}}
    \newcommand{\InformationTok}[1]{\textcolor[rgb]{0.38,0.63,0.69}{\textbf{\textit{{#1}}}}}
    \newcommand{\WarningTok}[1]{\textcolor[rgb]{0.38,0.63,0.69}{\textbf{\textit{{#1}}}}}


    % Define a nice break command that doesn't care if a line doesn't already
    % exist.
    \def\br{\hspace*{\fill} \\* }
    % Math Jax compatibility definitions
    \def\gt{>}
    \def\lt{<}
    \let\Oldtex\TeX
    \let\Oldlatex\LaTeX
    \renewcommand{\TeX}{\textrm{\Oldtex}}
    \renewcommand{\LaTeX}{\textrm{\Oldlatex}}
    % Document parameters
    % Document title
    \title{Notebook}
    
    
    
    
    
    
    
% Pygments definitions
\makeatletter
\def\PY@reset{\let\PY@it=\relax \let\PY@bf=\relax%
    \let\PY@ul=\relax \let\PY@tc=\relax%
    \let\PY@bc=\relax \let\PY@ff=\relax}
\def\PY@tok#1{\csname PY@tok@#1\endcsname}
\def\PY@toks#1+{\ifx\relax#1\empty\else%
    \PY@tok{#1}\expandafter\PY@toks\fi}
\def\PY@do#1{\PY@bc{\PY@tc{\PY@ul{%
    \PY@it{\PY@bf{\PY@ff{#1}}}}}}}
\def\PY#1#2{\PY@reset\PY@toks#1+\relax+\PY@do{#2}}

\@namedef{PY@tok@w}{\def\PY@tc##1{\textcolor[rgb]{0.73,0.73,0.73}{##1}}}
\@namedef{PY@tok@c}{\let\PY@it=\textit\def\PY@tc##1{\textcolor[rgb]{0.24,0.48,0.48}{##1}}}
\@namedef{PY@tok@cp}{\def\PY@tc##1{\textcolor[rgb]{0.61,0.40,0.00}{##1}}}
\@namedef{PY@tok@k}{\let\PY@bf=\textbf\def\PY@tc##1{\textcolor[rgb]{0.00,0.50,0.00}{##1}}}
\@namedef{PY@tok@kp}{\def\PY@tc##1{\textcolor[rgb]{0.00,0.50,0.00}{##1}}}
\@namedef{PY@tok@kt}{\def\PY@tc##1{\textcolor[rgb]{0.69,0.00,0.25}{##1}}}
\@namedef{PY@tok@o}{\def\PY@tc##1{\textcolor[rgb]{0.40,0.40,0.40}{##1}}}
\@namedef{PY@tok@ow}{\let\PY@bf=\textbf\def\PY@tc##1{\textcolor[rgb]{0.67,0.13,1.00}{##1}}}
\@namedef{PY@tok@nb}{\def\PY@tc##1{\textcolor[rgb]{0.00,0.50,0.00}{##1}}}
\@namedef{PY@tok@nf}{\def\PY@tc##1{\textcolor[rgb]{0.00,0.00,1.00}{##1}}}
\@namedef{PY@tok@nc}{\let\PY@bf=\textbf\def\PY@tc##1{\textcolor[rgb]{0.00,0.00,1.00}{##1}}}
\@namedef{PY@tok@nn}{\let\PY@bf=\textbf\def\PY@tc##1{\textcolor[rgb]{0.00,0.00,1.00}{##1}}}
\@namedef{PY@tok@ne}{\let\PY@bf=\textbf\def\PY@tc##1{\textcolor[rgb]{0.80,0.25,0.22}{##1}}}
\@namedef{PY@tok@nv}{\def\PY@tc##1{\textcolor[rgb]{0.10,0.09,0.49}{##1}}}
\@namedef{PY@tok@no}{\def\PY@tc##1{\textcolor[rgb]{0.53,0.00,0.00}{##1}}}
\@namedef{PY@tok@nl}{\def\PY@tc##1{\textcolor[rgb]{0.46,0.46,0.00}{##1}}}
\@namedef{PY@tok@ni}{\let\PY@bf=\textbf\def\PY@tc##1{\textcolor[rgb]{0.44,0.44,0.44}{##1}}}
\@namedef{PY@tok@na}{\def\PY@tc##1{\textcolor[rgb]{0.41,0.47,0.13}{##1}}}
\@namedef{PY@tok@nt}{\let\PY@bf=\textbf\def\PY@tc##1{\textcolor[rgb]{0.00,0.50,0.00}{##1}}}
\@namedef{PY@tok@nd}{\def\PY@tc##1{\textcolor[rgb]{0.67,0.13,1.00}{##1}}}
\@namedef{PY@tok@s}{\def\PY@tc##1{\textcolor[rgb]{0.73,0.13,0.13}{##1}}}
\@namedef{PY@tok@sd}{\let\PY@it=\textit\def\PY@tc##1{\textcolor[rgb]{0.73,0.13,0.13}{##1}}}
\@namedef{PY@tok@si}{\let\PY@bf=\textbf\def\PY@tc##1{\textcolor[rgb]{0.64,0.35,0.47}{##1}}}
\@namedef{PY@tok@se}{\let\PY@bf=\textbf\def\PY@tc##1{\textcolor[rgb]{0.67,0.36,0.12}{##1}}}
\@namedef{PY@tok@sr}{\def\PY@tc##1{\textcolor[rgb]{0.64,0.35,0.47}{##1}}}
\@namedef{PY@tok@ss}{\def\PY@tc##1{\textcolor[rgb]{0.10,0.09,0.49}{##1}}}
\@namedef{PY@tok@sx}{\def\PY@tc##1{\textcolor[rgb]{0.00,0.50,0.00}{##1}}}
\@namedef{PY@tok@m}{\def\PY@tc##1{\textcolor[rgb]{0.40,0.40,0.40}{##1}}}
\@namedef{PY@tok@gh}{\let\PY@bf=\textbf\def\PY@tc##1{\textcolor[rgb]{0.00,0.00,0.50}{##1}}}
\@namedef{PY@tok@gu}{\let\PY@bf=\textbf\def\PY@tc##1{\textcolor[rgb]{0.50,0.00,0.50}{##1}}}
\@namedef{PY@tok@gd}{\def\PY@tc##1{\textcolor[rgb]{0.63,0.00,0.00}{##1}}}
\@namedef{PY@tok@gi}{\def\PY@tc##1{\textcolor[rgb]{0.00,0.52,0.00}{##1}}}
\@namedef{PY@tok@gr}{\def\PY@tc##1{\textcolor[rgb]{0.89,0.00,0.00}{##1}}}
\@namedef{PY@tok@ge}{\let\PY@it=\textit}
\@namedef{PY@tok@gs}{\let\PY@bf=\textbf}
\@namedef{PY@tok@gp}{\let\PY@bf=\textbf\def\PY@tc##1{\textcolor[rgb]{0.00,0.00,0.50}{##1}}}
\@namedef{PY@tok@go}{\def\PY@tc##1{\textcolor[rgb]{0.44,0.44,0.44}{##1}}}
\@namedef{PY@tok@gt}{\def\PY@tc##1{\textcolor[rgb]{0.00,0.27,0.87}{##1}}}
\@namedef{PY@tok@err}{\def\PY@bc##1{{\setlength{\fboxsep}{\string -\fboxrule}\fcolorbox[rgb]{1.00,0.00,0.00}{1,1,1}{\strut ##1}}}}
\@namedef{PY@tok@kc}{\let\PY@bf=\textbf\def\PY@tc##1{\textcolor[rgb]{0.00,0.50,0.00}{##1}}}
\@namedef{PY@tok@kd}{\let\PY@bf=\textbf\def\PY@tc##1{\textcolor[rgb]{0.00,0.50,0.00}{##1}}}
\@namedef{PY@tok@kn}{\let\PY@bf=\textbf\def\PY@tc##1{\textcolor[rgb]{0.00,0.50,0.00}{##1}}}
\@namedef{PY@tok@kr}{\let\PY@bf=\textbf\def\PY@tc##1{\textcolor[rgb]{0.00,0.50,0.00}{##1}}}
\@namedef{PY@tok@bp}{\def\PY@tc##1{\textcolor[rgb]{0.00,0.50,0.00}{##1}}}
\@namedef{PY@tok@fm}{\def\PY@tc##1{\textcolor[rgb]{0.00,0.00,1.00}{##1}}}
\@namedef{PY@tok@vc}{\def\PY@tc##1{\textcolor[rgb]{0.10,0.09,0.49}{##1}}}
\@namedef{PY@tok@vg}{\def\PY@tc##1{\textcolor[rgb]{0.10,0.09,0.49}{##1}}}
\@namedef{PY@tok@vi}{\def\PY@tc##1{\textcolor[rgb]{0.10,0.09,0.49}{##1}}}
\@namedef{PY@tok@vm}{\def\PY@tc##1{\textcolor[rgb]{0.10,0.09,0.49}{##1}}}
\@namedef{PY@tok@sa}{\def\PY@tc##1{\textcolor[rgb]{0.73,0.13,0.13}{##1}}}
\@namedef{PY@tok@sb}{\def\PY@tc##1{\textcolor[rgb]{0.73,0.13,0.13}{##1}}}
\@namedef{PY@tok@sc}{\def\PY@tc##1{\textcolor[rgb]{0.73,0.13,0.13}{##1}}}
\@namedef{PY@tok@dl}{\def\PY@tc##1{\textcolor[rgb]{0.73,0.13,0.13}{##1}}}
\@namedef{PY@tok@s2}{\def\PY@tc##1{\textcolor[rgb]{0.73,0.13,0.13}{##1}}}
\@namedef{PY@tok@sh}{\def\PY@tc##1{\textcolor[rgb]{0.73,0.13,0.13}{##1}}}
\@namedef{PY@tok@s1}{\def\PY@tc##1{\textcolor[rgb]{0.73,0.13,0.13}{##1}}}
\@namedef{PY@tok@mb}{\def\PY@tc##1{\textcolor[rgb]{0.40,0.40,0.40}{##1}}}
\@namedef{PY@tok@mf}{\def\PY@tc##1{\textcolor[rgb]{0.40,0.40,0.40}{##1}}}
\@namedef{PY@tok@mh}{\def\PY@tc##1{\textcolor[rgb]{0.40,0.40,0.40}{##1}}}
\@namedef{PY@tok@mi}{\def\PY@tc##1{\textcolor[rgb]{0.40,0.40,0.40}{##1}}}
\@namedef{PY@tok@il}{\def\PY@tc##1{\textcolor[rgb]{0.40,0.40,0.40}{##1}}}
\@namedef{PY@tok@mo}{\def\PY@tc##1{\textcolor[rgb]{0.40,0.40,0.40}{##1}}}
\@namedef{PY@tok@ch}{\let\PY@it=\textit\def\PY@tc##1{\textcolor[rgb]{0.24,0.48,0.48}{##1}}}
\@namedef{PY@tok@cm}{\let\PY@it=\textit\def\PY@tc##1{\textcolor[rgb]{0.24,0.48,0.48}{##1}}}
\@namedef{PY@tok@cpf}{\let\PY@it=\textit\def\PY@tc##1{\textcolor[rgb]{0.24,0.48,0.48}{##1}}}
\@namedef{PY@tok@c1}{\let\PY@it=\textit\def\PY@tc##1{\textcolor[rgb]{0.24,0.48,0.48}{##1}}}
\@namedef{PY@tok@cs}{\let\PY@it=\textit\def\PY@tc##1{\textcolor[rgb]{0.24,0.48,0.48}{##1}}}

\def\PYZbs{\char`\\}
\def\PYZus{\char`\_}
\def\PYZob{\char`\{}
\def\PYZcb{\char`\}}
\def\PYZca{\char`\^}
\def\PYZam{\char`\&}
\def\PYZlt{\char`\<}
\def\PYZgt{\char`\>}
\def\PYZsh{\char`\#}
\def\PYZpc{\char`\%}
\def\PYZdl{\char`\$}
\def\PYZhy{\char`\-}
\def\PYZsq{\char`\'}
\def\PYZdq{\char`\"}
\def\PYZti{\char`\~}
% for compatibility with earlier versions
\def\PYZat{@}
\def\PYZlb{[}
\def\PYZrb{]}
\makeatother


    % For linebreaks inside Verbatim environment from package fancyvrb.
    \makeatletter
        \newbox\Wrappedcontinuationbox
        \newbox\Wrappedvisiblespacebox
        \newcommand*\Wrappedvisiblespace {\textcolor{red}{\textvisiblespace}}
        \newcommand*\Wrappedcontinuationsymbol {\textcolor{red}{\llap{\tiny$\m@th\hookrightarrow$}}}
        \newcommand*\Wrappedcontinuationindent {3ex }
        \newcommand*\Wrappedafterbreak {\kern\Wrappedcontinuationindent\copy\Wrappedcontinuationbox}
        % Take advantage of the already applied Pygments mark-up to insert
        % potential linebreaks for TeX processing.
        %        {, <, #, %, $, ' and ": go to next line.
        %        _, }, ^, &, >, - and ~: stay at end of broken line.
        % Use of \textquotesingle for straight quote.
        \newcommand*\Wrappedbreaksatspecials {%
            \def\PYGZus{\discretionary{\char`\_}{\Wrappedafterbreak}{\char`\_}}%
            \def\PYGZob{\discretionary{}{\Wrappedafterbreak\char`\{}{\char`\{}}%
            \def\PYGZcb{\discretionary{\char`\}}{\Wrappedafterbreak}{\char`\}}}%
            \def\PYGZca{\discretionary{\char`\^}{\Wrappedafterbreak}{\char`\^}}%
            \def\PYGZam{\discretionary{\char`\&}{\Wrappedafterbreak}{\char`\&}}%
            \def\PYGZlt{\discretionary{}{\Wrappedafterbreak\char`\<}{\char`\<}}%
            \def\PYGZgt{\discretionary{\char`\>}{\Wrappedafterbreak}{\char`\>}}%
            \def\PYGZsh{\discretionary{}{\Wrappedafterbreak\char`\#}{\char`\#}}%
            \def\PYGZpc{\discretionary{}{\Wrappedafterbreak\char`\%}{\char`\%}}%
            \def\PYGZdl{\discretionary{}{\Wrappedafterbreak\char`\$}{\char`\$}}%
            \def\PYGZhy{\discretionary{\char`\-}{\Wrappedafterbreak}{\char`\-}}%
            \def\PYGZsq{\discretionary{}{\Wrappedafterbreak\textquotesingle}{\textquotesingle}}%
            \def\PYGZdq{\discretionary{}{\Wrappedafterbreak\char`\"}{\char`\"}}%
            \def\PYGZti{\discretionary{\char`\~}{\Wrappedafterbreak}{\char`\~}}%
        }
        % Some characters . , ; ? ! / are not pygmentized.
        % This macro makes them "active" and they will insert potential linebreaks
        \newcommand*\Wrappedbreaksatpunct {%
            \lccode`\~`\.\lowercase{\def~}{\discretionary{\hbox{\char`\.}}{\Wrappedafterbreak}{\hbox{\char`\.}}}%
            \lccode`\~`\,\lowercase{\def~}{\discretionary{\hbox{\char`\,}}{\Wrappedafterbreak}{\hbox{\char`\,}}}%
            \lccode`\~`\;\lowercase{\def~}{\discretionary{\hbox{\char`\;}}{\Wrappedafterbreak}{\hbox{\char`\;}}}%
            \lccode`\~`\:\lowercase{\def~}{\discretionary{\hbox{\char`\:}}{\Wrappedafterbreak}{\hbox{\char`\:}}}%
            \lccode`\~`\?\lowercase{\def~}{\discretionary{\hbox{\char`\?}}{\Wrappedafterbreak}{\hbox{\char`\?}}}%
            \lccode`\~`\!\lowercase{\def~}{\discretionary{\hbox{\char`\!}}{\Wrappedafterbreak}{\hbox{\char`\!}}}%
            \lccode`\~`\/\lowercase{\def~}{\discretionary{\hbox{\char`\/}}{\Wrappedafterbreak}{\hbox{\char`\/}}}%
            \catcode`\.\active
            \catcode`\,\active
            \catcode`\;\active
            \catcode`\:\active
            \catcode`\?\active
            \catcode`\!\active
            \catcode`\/\active
            \lccode`\~`\~
        }
    \makeatother

    \let\OriginalVerbatim=\Verbatim
    \makeatletter
    \renewcommand{\Verbatim}[1][1]{%
        %\parskip\z@skip
        \sbox\Wrappedcontinuationbox {\Wrappedcontinuationsymbol}%
        \sbox\Wrappedvisiblespacebox {\FV@SetupFont\Wrappedvisiblespace}%
        \def\FancyVerbFormatLine ##1{\hsize\linewidth
            \vtop{\raggedright\hyphenpenalty\z@\exhyphenpenalty\z@
                \doublehyphendemerits\z@\finalhyphendemerits\z@
                \strut ##1\strut}%
        }%
        % If the linebreak is at a space, the latter will be displayed as visible
        % space at end of first line, and a continuation symbol starts next line.
        % Stretch/shrink are however usually zero for typewriter font.
        \def\FV@Space {%
            \nobreak\hskip\z@ plus\fontdimen3\font minus\fontdimen4\font
            \discretionary{\copy\Wrappedvisiblespacebox}{\Wrappedafterbreak}
            {\kern\fontdimen2\font}%
        }%

        % Allow breaks at special characters using \PYG... macros.
        \Wrappedbreaksatspecials
        % Breaks at punctuation characters . , ; ? ! and / need catcode=\active
        \OriginalVerbatim[#1,codes*=\Wrappedbreaksatpunct]%
    }
    \makeatother

    % Exact colors from NB
    \definecolor{incolor}{HTML}{303F9F}
    \definecolor{outcolor}{HTML}{D84315}
    \definecolor{cellborder}{HTML}{CFCFCF}
    \definecolor{cellbackground}{HTML}{F7F7F7}

    % prompt
    \makeatletter
    \newcommand{\boxspacing}{\kern\kvtcb@left@rule\kern\kvtcb@boxsep}
    \makeatother
    \newcommand{\prompt}[4]{
        {\ttfamily\llap{{\color{#2}[#3]:\hspace{3pt}#4}}\vspace{-\baselineskip}}
    }
    

    
    % Prevent overflowing lines due to hard-to-break entities
    \sloppy
    % Setup hyperref package
    \hypersetup{
      breaklinks=true,  % so long urls are correctly broken across lines
      colorlinks=true,
      urlcolor=urlcolor,
      linkcolor=linkcolor,
      citecolor=citecolor,
      }
    % Slightly bigger margins than the latex defaults
    
    \geometry{verbose,tmargin=1in,bmargin=1in,lmargin=1in,rmargin=1in}
    
    

\begin{document}
    
    \maketitle
    
    

    
    \hypertarget{exercise-supervised-learning}{%
\section{Exercise: Supervised
learning}\label{exercise-supervised-learning}}

Recall our farming scenario, in which we want to look at how January
temperatures have changed over time. Now we'll build a model that
achieves this by using supervised learning.

With many libraries, we can build a model in only a few lines of code.
Here, we'll break down the process into steps so that we can explore how
things work.

\hypertarget{four-components}{%
\subsection{Four components}\label{four-components}}

Recall that there are four key components to supervised learning: the
data, the model, the cost function, and the optimizer. Let's inspect
these one at a time.

\hypertarget{the-data}{%
\subsubsection{1. The data}\label{the-data}}

We'll use publicly available weather data for Seattle. Let's load that
and restrict it to January temperatures.

    \begin{tcolorbox}[breakable, size=fbox, boxrule=1pt, pad at break*=1mm,colback=cellbackground, colframe=cellborder]
\prompt{In}{incolor}{1}{\boxspacing}
\begin{Verbatim}[commandchars=\\\{\}]
\PY{k+kn}{import} \PY{n+nn}{pandas}
\PY{o}{!}wget\PY{+w}{ }https://raw.githubusercontent.com/MicrosoftDocs/mslearn\PYZhy{}introduction\PYZhy{}to\PYZhy{}machine\PYZhy{}learning/main/graphing.py
\PY{o}{!}wget\PY{+w}{ }https://raw.githubusercontent.com/MicrosoftDocs/mslearn\PYZhy{}introduction\PYZhy{}to\PYZhy{}machine\PYZhy{}learning/main/m0b\PYZus{}optimizer.py
\PY{o}{!}wget\PY{+w}{ }https://raw.githubusercontent.com/MicrosoftDocs/mslearn\PYZhy{}introduction\PYZhy{}to\PYZhy{}machine\PYZhy{}learning/main/Data/seattleWeather\PYZus{}1948\PYZhy{}2017.csv

\PY{c+c1}{\PYZsh{} Load a file that contains weather data for Seattle}
\PY{n}{data} \PY{o}{=} \PY{n}{pandas}\PY{o}{.}\PY{n}{read\PYZus{}csv}\PY{p}{(}\PY{l+s+s1}{\PYZsq{}}\PY{l+s+s1}{seattleWeather\PYZus{}1948\PYZhy{}2017.csv}\PY{l+s+s1}{\PYZsq{}}\PY{p}{,} \PY{n}{parse\PYZus{}dates}\PY{o}{=}\PY{p}{[}\PY{l+s+s1}{\PYZsq{}}\PY{l+s+s1}{date}\PY{l+s+s1}{\PYZsq{}}\PY{p}{]}\PY{p}{)}

\PY{c+c1}{\PYZsh{} Keep only January temperatures}
\PY{n}{data} \PY{o}{=} \PY{n}{data}\PY{p}{[}\PY{p}{[}\PY{n}{d}\PY{o}{.}\PY{n}{month} \PY{o}{==} \PY{l+m+mi}{1} \PY{k}{for} \PY{n}{d} \PY{o+ow}{in} \PY{n}{data}\PY{o}{.}\PY{n}{date}\PY{p}{]}\PY{p}{]}\PY{o}{.}\PY{n}{copy}\PY{p}{(}\PY{p}{)}


\PY{c+c1}{\PYZsh{} Print the first and last few rows}
\PY{c+c1}{\PYZsh{} Remember that with Jupyter notebooks, the last line of }
\PY{c+c1}{\PYZsh{} code is automatically printed}
\PY{n}{data}
\end{Verbatim}
\end{tcolorbox}

    \begin{Verbatim}[commandchars=\\\{\}]
--2023-08-18 11:11:45--
https://raw.githubusercontent.com/MicrosoftDocs/mslearn-introduction-to-machine-
learning/main/graphing.py
Resolving raw.githubusercontent.com (raw.githubusercontent.com){\ldots}
185.199.111.133, 185.199.110.133, 185.199.109.133, {\ldots}
Connecting to raw.githubusercontent.com
(raw.githubusercontent.com)|185.199.111.133|:443{\ldots} connected.
HTTP request sent, awaiting response{\ldots} 200 OK
Length: 21511 (21K) [text/plain]
Saving to: ‘graphing.py’

graphing.py         100\%[===================>]  21.01K  --.-KB/s    in 0.001s

2023-08-18 11:11:45 (35.7 MB/s) - ‘graphing.py’ saved [21511/21511]

--2023-08-18 11:11:47--
https://raw.githubusercontent.com/MicrosoftDocs/mslearn-introduction-to-machine-
learning/main/m0b\_optimizer.py
Resolving raw.githubusercontent.com (raw.githubusercontent.com){\ldots}
185.199.108.133, 185.199.109.133, 185.199.110.133, {\ldots}
Connecting to raw.githubusercontent.com
(raw.githubusercontent.com)|185.199.108.133|:443{\ldots} connected.
HTTP request sent, awaiting response{\ldots} 200 OK
Length: 1287 (1.3K) [text/plain]
Saving to: ‘m0b\_optimizer.py’

m0b\_optimizer.py    100\%[===================>]   1.26K  --.-KB/s    in 0s

2023-08-18 11:11:47 (60.5 MB/s) - ‘m0b\_optimizer.py’ saved [1287/1287]

--2023-08-18 11:11:49--
https://raw.githubusercontent.com/MicrosoftDocs/mslearn-introduction-to-machine-
learning/main/Data/seattleWeather\_1948-2017.csv
Resolving raw.githubusercontent.com (raw.githubusercontent.com){\ldots}
185.199.108.133, 185.199.109.133, 185.199.110.133, {\ldots}
Connecting to raw.githubusercontent.com
(raw.githubusercontent.com)|185.199.108.133|:443{\ldots} connected.
HTTP request sent, awaiting response{\ldots} 200 OK
Length: 762017 (744K) [text/plain]
Saving to: ‘seattleWeather\_1948-2017.csv’

seattleWeather\_1948 100\%[===================>] 744.16K  --.-KB/s    in 0.008s

2023-08-18 11:11:49 (92.9 MB/s) - ‘seattleWeather\_1948-2017.csv’ saved
[762017/762017]

    \end{Verbatim}

            \begin{tcolorbox}[breakable, size=fbox, boxrule=.5pt, pad at break*=1mm, opacityfill=0]
\prompt{Out}{outcolor}{1}{\boxspacing}
\begin{Verbatim}[commandchars=\\\{\}]
            date  amount\_of\_precipitation  max\_temperature  min\_temperature  \textbackslash{}
0     1948-01-01                     0.47               51               42
1     1948-01-02                     0.59               45               36
2     1948-01-03                     0.42               45               35
3     1948-01-04                     0.31               45               34
4     1948-01-05                     0.17               45               32
{\ldots}          {\ldots}                      {\ldots}              {\ldots}              {\ldots}
25229 2017-01-27                     0.00               54               37
25230 2017-01-28                     0.00               52               37
25231 2017-01-29                     0.03               48               37
25232 2017-01-30                     0.02               45               40
25233 2017-01-31                     0.00               44               34

        rain
0       True
1       True
2       True
3       True
4       True
{\ldots}      {\ldots}
25229  False
25230  False
25231   True
25232   True
25233  False

[2170 rows x 5 columns]
\end{Verbatim}
\end{tcolorbox}
        
    We have data from 1948 to 2017, split across 2,170 rows.

We'll analyze the relationship between \texttt{date} and daily minimum
temperatures. Let's take a quick look at our data as a graph.

    \begin{tcolorbox}[breakable, size=fbox, boxrule=1pt, pad at break*=1mm,colback=cellbackground, colframe=cellborder]
\prompt{In}{incolor}{2}{\boxspacing}
\begin{Verbatim}[commandchars=\\\{\}]
\PY{k+kn}{import} \PY{n+nn}{graphing} \PY{c+c1}{\PYZsh{} Custom graphing code. See our GitHub repository for details}

\PY{c+c1}{\PYZsh{} Let\PYZsq{}s take a quick look at our data}
\PY{n}{graphing}\PY{o}{.}\PY{n}{scatter\PYZus{}2D}\PY{p}{(}\PY{n}{data}\PY{p}{,} \PY{n}{label\PYZus{}x}\PY{o}{=}\PY{l+s+s2}{\PYZdq{}}\PY{l+s+s2}{date}\PY{l+s+s2}{\PYZdq{}}\PY{p}{,} \PY{n}{label\PYZus{}y}\PY{o}{=}\PY{l+s+s2}{\PYZdq{}}\PY{l+s+s2}{min\PYZus{}temperature}\PY{l+s+s2}{\PYZdq{}}\PY{p}{,} \PY{n}{title}\PY{o}{=}\PY{l+s+s2}{\PYZdq{}}\PY{l+s+s2}{January Temperatures (°F)}\PY{l+s+s2}{\PYZdq{}}\PY{p}{)}
\end{Verbatim}
\end{tcolorbox}

    
    
    
    
    Machine learning usually works best when the X and Y axes have roughly
the same range of values. We'll cover why in later learning material.
For now, let's just scale our data slightly.

    \begin{tcolorbox}[breakable, size=fbox, boxrule=1pt, pad at break*=1mm,colback=cellbackground, colframe=cellborder]
\prompt{In}{incolor}{3}{\boxspacing}
\begin{Verbatim}[commandchars=\\\{\}]
\PY{k+kn}{import} \PY{n+nn}{numpy} \PY{k}{as} \PY{n+nn}{np}

\PY{c+c1}{\PYZsh{} This block of code scales and offsets the data slightly, which helps the training process}
\PY{c+c1}{\PYZsh{} You don\PYZsq{}t need to understand this code. We\PYZsq{}ll cover these concepts in later learning material}

\PY{c+c1}{\PYZsh{} Offset date into number of years since 1982}
\PY{n}{data}\PY{p}{[}\PY{l+s+s2}{\PYZdq{}}\PY{l+s+s2}{years\PYZus{}since\PYZus{}1982}\PY{l+s+s2}{\PYZdq{}}\PY{p}{]} \PY{o}{=} \PY{p}{[}\PY{p}{(}\PY{n}{d}\PY{o}{.}\PY{n}{year} \PY{o}{+} \PY{n}{d}\PY{o}{.}\PY{n}{timetuple}\PY{p}{(}\PY{p}{)}\PY{o}{.}\PY{n}{tm\PYZus{}yday} \PY{o}{/} \PY{l+m+mf}{365.25}\PY{p}{)} \PY{o}{\PYZhy{}} \PY{l+m+mi}{1982} \PY{k}{for} \PY{n}{d} \PY{o+ow}{in} \PY{n}{data}\PY{o}{.}\PY{n}{date}\PY{p}{]}

\PY{c+c1}{\PYZsh{} Scale and offset temperature so that it has a smaller range of values}
\PY{n}{data}\PY{p}{[}\PY{l+s+s2}{\PYZdq{}}\PY{l+s+s2}{normalised\PYZus{}temperature}\PY{l+s+s2}{\PYZdq{}}\PY{p}{]} \PY{o}{=} \PY{p}{(}\PY{n}{data}\PY{p}{[}\PY{l+s+s2}{\PYZdq{}}\PY{l+s+s2}{min\PYZus{}temperature}\PY{l+s+s2}{\PYZdq{}}\PY{p}{]} \PY{o}{\PYZhy{}} \PY{n}{np}\PY{o}{.}\PY{n}{mean}\PY{p}{(}\PY{n}{data}\PY{p}{[}\PY{l+s+s2}{\PYZdq{}}\PY{l+s+s2}{min\PYZus{}temperature}\PY{l+s+s2}{\PYZdq{}}\PY{p}{]}\PY{p}{)}\PY{p}{)} \PY{o}{/} \PY{n}{np}\PY{o}{.}\PY{n}{std}\PY{p}{(}\PY{n}{data}\PY{p}{[}\PY{l+s+s2}{\PYZdq{}}\PY{l+s+s2}{min\PYZus{}temperature}\PY{l+s+s2}{\PYZdq{}}\PY{p}{]}\PY{p}{)}

\PY{c+c1}{\PYZsh{} Graph}
\PY{n}{graphing}\PY{o}{.}\PY{n}{scatter\PYZus{}2D}\PY{p}{(}\PY{n}{data}\PY{p}{,} \PY{n}{label\PYZus{}x}\PY{o}{=}\PY{l+s+s2}{\PYZdq{}}\PY{l+s+s2}{years\PYZus{}since\PYZus{}1982}\PY{l+s+s2}{\PYZdq{}}\PY{p}{,} \PY{n}{label\PYZus{}y}\PY{o}{=}\PY{l+s+s2}{\PYZdq{}}\PY{l+s+s2}{normalised\PYZus{}temperature}\PY{l+s+s2}{\PYZdq{}}\PY{p}{,} \PY{n}{title}\PY{o}{=}\PY{l+s+s2}{\PYZdq{}}\PY{l+s+s2}{January Temperatures (Normalised)}\PY{l+s+s2}{\PYZdq{}}\PY{p}{)}
\end{Verbatim}
\end{tcolorbox}

    
    
    \hypertarget{the-model}{%
\subsubsection{2. The model}\label{the-model}}

We'll select a simple linear regression model. This model uses a line to
make estimates. You might have come across trendlines like these before
when making graphs.

    \begin{tcolorbox}[breakable, size=fbox, boxrule=1pt, pad at break*=1mm,colback=cellbackground, colframe=cellborder]
\prompt{In}{incolor}{4}{\boxspacing}
\begin{Verbatim}[commandchars=\\\{\}]
\PY{k}{class} \PY{n+nc}{MyModel}\PY{p}{:}

    \PY{k}{def} \PY{n+nf+fm}{\PYZus{}\PYZus{}init\PYZus{}\PYZus{}}\PY{p}{(}\PY{n+nb+bp}{self}\PY{p}{)}\PY{p}{:}
\PY{+w}{        }\PY{l+s+sd}{\PYZsq{}\PYZsq{}\PYZsq{}}
\PY{l+s+sd}{        Creates a new MyModel}
\PY{l+s+sd}{        \PYZsq{}\PYZsq{}\PYZsq{}}
        \PY{c+c1}{\PYZsh{} Straight lines described by two parameters:}
        \PY{c+c1}{\PYZsh{} The slope is the angle of the line}
        \PY{n+nb+bp}{self}\PY{o}{.}\PY{n}{slope} \PY{o}{=} \PY{l+m+mi}{0}
        \PY{c+c1}{\PYZsh{} The intercept moves the line up or down}
        \PY{n+nb+bp}{self}\PY{o}{.}\PY{n}{intercept} \PY{o}{=} \PY{l+m+mi}{0}

    \PY{k}{def} \PY{n+nf}{predict}\PY{p}{(}\PY{n+nb+bp}{self}\PY{p}{,} \PY{n}{date}\PY{p}{)}\PY{p}{:}
\PY{+w}{        }\PY{l+s+sd}{\PYZsq{}\PYZsq{}\PYZsq{}}
\PY{l+s+sd}{        Estimates the temperature from the date}
\PY{l+s+sd}{        \PYZsq{}\PYZsq{}\PYZsq{}}
        \PY{k}{return} \PY{n}{date} \PY{o}{*} \PY{n+nb+bp}{self}\PY{o}{.}\PY{n}{slope} \PY{o}{+} \PY{n+nb+bp}{self}\PY{o}{.}\PY{n}{intercept}

\PY{c+c1}{\PYZsh{} Create our model ready to be trained}
\PY{n}{model} \PY{o}{=} \PY{n}{MyModel}\PY{p}{(}\PY{p}{)}

\PY{n+nb}{print}\PY{p}{(}\PY{l+s+s2}{\PYZdq{}}\PY{l+s+s2}{Model made!}\PY{l+s+s2}{\PYZdq{}}\PY{p}{)}
\end{Verbatim}
\end{tcolorbox}

    \begin{Verbatim}[commandchars=\\\{\}]
Model made!
    \end{Verbatim}

    We wouldn't normally use a model before it has been trained, but for the
sake of learning, let's take a quick look at it.

    \begin{tcolorbox}[breakable, size=fbox, boxrule=1pt, pad at break*=1mm,colback=cellbackground, colframe=cellborder]
\prompt{In}{incolor}{5}{\boxspacing}
\begin{Verbatim}[commandchars=\\\{\}]
\PY{n+nb}{print}\PY{p}{(}\PY{l+s+sa}{f}\PY{l+s+s2}{\PYZdq{}}\PY{l+s+s2}{Model parameters before training: }\PY{l+s+si}{\PYZob{}}\PY{n}{model}\PY{o}{.}\PY{n}{intercept}\PY{l+s+si}{\PYZcb{}}\PY{l+s+s2}{, }\PY{l+s+si}{\PYZob{}}\PY{n}{model}\PY{o}{.}\PY{n}{slope}\PY{l+s+si}{\PYZcb{}}\PY{l+s+s2}{\PYZdq{}}\PY{p}{)}

\PY{c+c1}{\PYZsh{} Look at how well the model does before training}
\PY{n+nb}{print}\PY{p}{(}\PY{l+s+s2}{\PYZdq{}}\PY{l+s+s2}{Model visualised before training:}\PY{l+s+s2}{\PYZdq{}}\PY{p}{)}
\PY{n}{graphing}\PY{o}{.}\PY{n}{scatter\PYZus{}2D}\PY{p}{(}\PY{n}{data}\PY{p}{,} \PY{l+s+s2}{\PYZdq{}}\PY{l+s+s2}{years\PYZus{}since\PYZus{}1982}\PY{l+s+s2}{\PYZdq{}}\PY{p}{,} \PY{l+s+s2}{\PYZdq{}}\PY{l+s+s2}{normalised\PYZus{}temperature}\PY{l+s+s2}{\PYZdq{}}\PY{p}{,} \PY{n}{trendline}\PY{o}{=}\PY{n}{model}\PY{o}{.}\PY{n}{predict}\PY{p}{)}   
\end{Verbatim}
\end{tcolorbox}

    \begin{Verbatim}[commandchars=\\\{\}]
Model parameters before training: 0, 0
Model visualised before training:
    \end{Verbatim}

    
    
    You can see that before training, our model (the red line) isn't useful
at all. It always simply predicts zero.

\hypertarget{the-cost-objective-function}{%
\subsubsection{3. The cost (objective)
function}\label{the-cost-objective-function}}

Our next step is to create a \emph{cost function} (\emph{objective
function}).

These functions in supervised learning compare the model's estimate to
the correct answer. In our case, our label is temperature, so our cost
function will compare the estimated temperature to temperatures seen in
the historical records.

    \begin{tcolorbox}[breakable, size=fbox, boxrule=1pt, pad at break*=1mm,colback=cellbackground, colframe=cellborder]
\prompt{In}{incolor}{6}{\boxspacing}
\begin{Verbatim}[commandchars=\\\{\}]
\PY{k}{def} \PY{n+nf}{cost\PYZus{}function}\PY{p}{(}\PY{n}{actual\PYZus{}temperatures}\PY{p}{,} \PY{n}{estimated\PYZus{}temperatures}\PY{p}{)}\PY{p}{:}
\PY{+w}{    }\PY{l+s+sd}{\PYZsq{}\PYZsq{}\PYZsq{}}
\PY{l+s+sd}{    Calculates the difference between actual and estimated temperatures}
\PY{l+s+sd}{    Returns the difference, and also returns the squared difference (the cost)}

\PY{l+s+sd}{    actual\PYZus{}temperatures: One or more temperatures recorded in the past}
\PY{l+s+sd}{    estimated\PYZus{}temperatures: Corresponding temperature(s) estimated by the model}
\PY{l+s+sd}{    \PYZsq{}\PYZsq{}\PYZsq{}}

    \PY{c+c1}{\PYZsh{} Calculate the difference between actual temperatures and those}
    \PY{c+c1}{\PYZsh{} estimated by the model}
    \PY{n}{difference} \PY{o}{=} \PY{n}{estimated\PYZus{}temperatures} \PY{o}{\PYZhy{}} \PY{n}{actual\PYZus{}temperatures}

    \PY{c+c1}{\PYZsh{} Convert to a single number that tells us how well the model did}
    \PY{c+c1}{\PYZsh{} (smaller numbers are better)}
    \PY{n}{cost} \PY{o}{=} \PY{n+nb}{sum}\PY{p}{(}\PY{n}{difference} \PY{o}{*}\PY{o}{*} \PY{l+m+mi}{2}\PY{p}{)}

    \PY{k}{return} \PY{n}{difference}\PY{p}{,} \PY{n}{cost}
\end{Verbatim}
\end{tcolorbox}

    \hypertarget{the-optimizer}{%
\subsubsection{4. The optimizer}\label{the-optimizer}}

The role of the optimizer is to guess new parameter values for the
model.

We haven't covered optimizers in detail yet, so to make things simple,
we'll use an prewritten optimizer. You don't need to understand how this
works, but if you're curious, you can find it in our GitHub repository.

    \begin{tcolorbox}[breakable, size=fbox, boxrule=1pt, pad at break*=1mm,colback=cellbackground, colframe=cellborder]
\prompt{In}{incolor}{7}{\boxspacing}
\begin{Verbatim}[commandchars=\\\{\}]
\PY{k+kn}{from} \PY{n+nn}{m0b\PYZus{}optimizer} \PY{k+kn}{import} \PY{n}{MyOptimizer}

\PY{c+c1}{\PYZsh{} Create an optimizer}
\PY{n}{optimizer} \PY{o}{=} \PY{n}{MyOptimizer}\PY{p}{(}\PY{p}{)}
\end{Verbatim}
\end{tcolorbox}

    \hypertarget{the-training-loop}{%
\subsection{The training loop}\label{the-training-loop}}

Let's put these components together so that they train the model.

First, let's make a function that performs one iteration of training.
Read each step carefully in the following code. If you want, add some
\texttt{print()} statements inside the method to help you see the
training in action.

    \begin{tcolorbox}[breakable, size=fbox, boxrule=1pt, pad at break*=1mm,colback=cellbackground, colframe=cellborder]
\prompt{In}{incolor}{8}{\boxspacing}
\begin{Verbatim}[commandchars=\\\{\}]
\PY{k}{def} \PY{n+nf}{train\PYZus{}one\PYZus{}iteration}\PY{p}{(}\PY{n}{model\PYZus{}inputs}\PY{p}{,} \PY{n}{true\PYZus{}temperatures}\PY{p}{,} \PY{n}{last\PYZus{}cost}\PY{p}{:}\PY{n+nb}{float}\PY{p}{)}\PY{p}{:}
\PY{+w}{    }\PY{l+s+sd}{\PYZsq{}\PYZsq{}\PYZsq{}}
\PY{l+s+sd}{    Runs a single iteration of training.}


\PY{l+s+sd}{    model\PYZus{}inputs: One or more dates to provide the model (dates)}
\PY{l+s+sd}{    true\PYZus{}temperatues: Corresponding temperatures known to occur on those dates}

\PY{l+s+sd}{    Returns:}
\PY{l+s+sd}{        A Boolean, as to whether training should continue}
\PY{l+s+sd}{        The cost calculated (small numbers are better)}
\PY{l+s+sd}{    \PYZsq{}\PYZsq{}\PYZsq{}}

    \PY{c+c1}{\PYZsh{} === USE THE MODEL ===}
    \PY{c+c1}{\PYZsh{} Estimate temperatures for all data that we have}
    \PY{n}{estimated\PYZus{}temperatures} \PY{o}{=} \PY{n}{model}\PY{o}{.}\PY{n}{predict}\PY{p}{(}\PY{n}{model\PYZus{}inputs}\PY{p}{)}

    \PY{c+c1}{\PYZsh{} === OBJECTIVE FUNCTION ===}
    \PY{c+c1}{\PYZsh{} Calculate how well the model is working}
    \PY{c+c1}{\PYZsh{} Smaller numbers are better }
    \PY{n}{difference}\PY{p}{,} \PY{n}{cost} \PY{o}{=} \PY{n}{cost\PYZus{}function}\PY{p}{(}\PY{n}{true\PYZus{}temperatures}\PY{p}{,} \PY{n}{estimated\PYZus{}temperatures}\PY{p}{)}

    \PY{c+c1}{\PYZsh{} Decide whether to keep training}
    \PY{c+c1}{\PYZsh{} We\PYZsq{}ll stop if the training is no longer improving the model effectively}
    \PY{k}{if} \PY{n}{cost} \PY{o}{\PYZgt{}}\PY{o}{=} \PY{n}{last\PYZus{}cost}\PY{p}{:}
        \PY{c+c1}{\PYZsh{} Stop training}
        \PY{k}{return} \PY{k+kc}{False}\PY{p}{,} \PY{n}{cost}
    \PY{k}{else}\PY{p}{:}
        \PY{c+c1}{\PYZsh{} === OPTIMIZER ===}
        \PY{c+c1}{\PYZsh{} Calculate updates to parameters}
        \PY{n}{intercept\PYZus{}update}\PY{p}{,} \PY{n}{slope\PYZus{}update} \PY{o}{=} \PY{n}{optimizer}\PY{o}{.}\PY{n}{get\PYZus{}parameter\PYZus{}updates}\PY{p}{(}\PY{n}{model\PYZus{}inputs}\PY{p}{,} \PY{n}{cost}\PY{p}{,} \PY{n}{difference}\PY{p}{)}

        \PY{c+c1}{\PYZsh{} Change the model parameters}
        \PY{n}{model}\PY{o}{.}\PY{n}{slope} \PY{o}{+}\PY{o}{=} \PY{n}{slope\PYZus{}update}
        \PY{n}{model}\PY{o}{.}\PY{n}{intercept} \PY{o}{+}\PY{o}{=} \PY{n}{intercept\PYZus{}update}

        \PY{k}{return} \PY{k+kc}{True}\PY{p}{,} \PY{n}{cost}

\PY{n+nb}{print}\PY{p}{(}\PY{l+s+s2}{\PYZdq{}}\PY{l+s+s2}{Training method ready}\PY{l+s+s2}{\PYZdq{}}\PY{p}{)}
\end{Verbatim}
\end{tcolorbox}

    \begin{Verbatim}[commandchars=\\\{\}]
Training method ready
    \end{Verbatim}

    Let's run a few iterations manually, so that you can watch how training
works.

Run the following code several times, and note how the model changes.

    \begin{tcolorbox}[breakable, size=fbox, boxrule=1pt, pad at break*=1mm,colback=cellbackground, colframe=cellborder]
\prompt{In}{incolor}{11}{\boxspacing}
\begin{Verbatim}[commandchars=\\\{\}]
\PY{k+kn}{import} \PY{n+nn}{math}

\PY{n+nb}{print}\PY{p}{(}\PY{l+s+sa}{f}\PY{l+s+s2}{\PYZdq{}}\PY{l+s+s2}{Model parameters before training:}\PY{l+s+se}{\PYZbs{}t}\PY{l+s+se}{\PYZbs{}t}\PY{l+s+si}{\PYZob{}}\PY{n}{model}\PY{o}{.}\PY{n}{intercept}\PY{l+s+si}{:}\PY{l+s+s2}{.8f}\PY{l+s+si}{\PYZcb{}}\PY{l+s+s2}{,}\PY{l+s+se}{\PYZbs{}t}\PY{l+s+si}{\PYZob{}}\PY{n}{model}\PY{o}{.}\PY{n}{slope}\PY{l+s+si}{:}\PY{l+s+s2}{.8f}\PY{l+s+si}{\PYZcb{}}\PY{l+s+s2}{\PYZdq{}}\PY{p}{)}

\PY{n}{continue\PYZus{}loop}\PY{p}{,} \PY{n}{cost} \PY{o}{=} \PY{n}{train\PYZus{}one\PYZus{}iteration}\PY{p}{(}\PY{n}{model\PYZus{}inputs} \PY{o}{=} \PY{n}{data}\PY{p}{[}\PY{l+s+s2}{\PYZdq{}}\PY{l+s+s2}{years\PYZus{}since\PYZus{}1982}\PY{l+s+s2}{\PYZdq{}}\PY{p}{]}\PY{p}{,}
                                                    \PY{n}{true\PYZus{}temperatures} \PY{o}{=} \PY{n}{data}\PY{p}{[}\PY{l+s+s2}{\PYZdq{}}\PY{l+s+s2}{normalised\PYZus{}temperature}\PY{l+s+s2}{\PYZdq{}}\PY{p}{]}\PY{p}{,}
                                                    \PY{n}{last\PYZus{}cost} \PY{o}{=} \PY{n}{math}\PY{o}{.}\PY{n}{inf}\PY{p}{)}

\PY{n+nb}{print}\PY{p}{(}\PY{l+s+sa}{f}\PY{l+s+s2}{\PYZdq{}}\PY{l+s+s2}{Model parameters after 1 iteration of training:}\PY{l+s+se}{\PYZbs{}t}\PY{l+s+si}{\PYZob{}}\PY{n}{model}\PY{o}{.}\PY{n}{intercept}\PY{l+s+si}{:}\PY{l+s+s2}{.8f}\PY{l+s+si}{\PYZcb{}}\PY{l+s+s2}{,}\PY{l+s+se}{\PYZbs{}t}\PY{l+s+si}{\PYZob{}}\PY{n}{model}\PY{o}{.}\PY{n}{slope}\PY{l+s+si}{:}\PY{l+s+s2}{.8f}\PY{l+s+si}{\PYZcb{}}\PY{l+s+s2}{\PYZdq{}}\PY{p}{)}
\end{Verbatim}
\end{tcolorbox}

    \begin{Verbatim}[commandchars=\\\{\}]
Model parameters before training:               -0.00001132,    0.01163566
Model parameters after 1 iteration of training: -0.00002437,    0.01187966
    \end{Verbatim}

    It will take thousands of iterations to train the model well, so let's
wrap it in a loop.

    \begin{tcolorbox}[breakable, size=fbox, boxrule=1pt, pad at break*=1mm,colback=cellbackground, colframe=cellborder]
\prompt{In}{incolor}{12}{\boxspacing}
\begin{Verbatim}[commandchars=\\\{\}]
\PY{c+c1}{\PYZsh{} Start the loop}
\PY{n+nb}{print}\PY{p}{(}\PY{l+s+s2}{\PYZdq{}}\PY{l+s+s2}{Training beginning...}\PY{l+s+s2}{\PYZdq{}}\PY{p}{)}
\PY{n}{last\PYZus{}cost} \PY{o}{=} \PY{n}{math}\PY{o}{.}\PY{n}{inf}
\PY{n}{i} \PY{o}{=} \PY{l+m+mi}{0}
\PY{n}{continue\PYZus{}loop} \PY{o}{=} \PY{k+kc}{True}
\PY{k}{while} \PY{n}{continue\PYZus{}loop}\PY{p}{:}

    \PY{c+c1}{\PYZsh{} Run one iteration of training}
    \PY{c+c1}{\PYZsh{} This will tell us whether to stop training, and also what}
    \PY{c+c1}{\PYZsh{} the cost was for this iteration}
    \PY{n}{continue\PYZus{}loop}\PY{p}{,} \PY{n}{last\PYZus{}cost} \PY{o}{=} \PY{n}{train\PYZus{}one\PYZus{}iteration}\PY{p}{(}\PY{n}{model\PYZus{}inputs} \PY{o}{=} \PY{n}{data}\PY{p}{[}\PY{l+s+s2}{\PYZdq{}}\PY{l+s+s2}{years\PYZus{}since\PYZus{}1982}\PY{l+s+s2}{\PYZdq{}}\PY{p}{]}\PY{p}{,}
                                                    \PY{n}{true\PYZus{}temperatures} \PY{o}{=} \PY{n}{data}\PY{p}{[}\PY{l+s+s2}{\PYZdq{}}\PY{l+s+s2}{normalised\PYZus{}temperature}\PY{l+s+s2}{\PYZdq{}}\PY{p}{]}\PY{p}{,}
                                                    \PY{n}{last\PYZus{}cost} \PY{o}{=} \PY{n}{last\PYZus{}cost}\PY{p}{)}
   
    \PY{c+c1}{\PYZsh{} Print the status}
    \PY{k}{if} \PY{n}{i} \PY{o}{\PYZpc{}} \PY{l+m+mi}{400} \PY{o}{==} \PY{l+m+mi}{0}\PY{p}{:}
        \PY{n+nb}{print}\PY{p}{(}\PY{l+s+s2}{\PYZdq{}}\PY{l+s+s2}{Iteration:}\PY{l+s+s2}{\PYZdq{}}\PY{p}{,} \PY{n}{i}\PY{p}{)}

    \PY{n}{i} \PY{o}{+}\PY{o}{=} \PY{l+m+mi}{1}

    
\PY{n+nb}{print}\PY{p}{(}\PY{l+s+s2}{\PYZdq{}}\PY{l+s+s2}{Training complete!}\PY{l+s+s2}{\PYZdq{}}\PY{p}{)}
\PY{n+nb}{print}\PY{p}{(}\PY{l+s+sa}{f}\PY{l+s+s2}{\PYZdq{}}\PY{l+s+s2}{Model parameters after training:}\PY{l+s+se}{\PYZbs{}t}\PY{l+s+si}{\PYZob{}}\PY{n}{model}\PY{o}{.}\PY{n}{intercept}\PY{l+s+si}{:}\PY{l+s+s2}{.8f}\PY{l+s+si}{\PYZcb{}}\PY{l+s+s2}{,}\PY{l+s+se}{\PYZbs{}t}\PY{l+s+si}{\PYZob{}}\PY{n}{model}\PY{o}{.}\PY{n}{slope}\PY{l+s+si}{:}\PY{l+s+s2}{.8f}\PY{l+s+si}{\PYZcb{}}\PY{l+s+s2}{\PYZdq{}}\PY{p}{)}
\PY{n}{graphing}\PY{o}{.}\PY{n}{scatter\PYZus{}2D}\PY{p}{(}\PY{n}{data}\PY{p}{,} \PY{l+s+s2}{\PYZdq{}}\PY{l+s+s2}{years\PYZus{}since\PYZus{}1982}\PY{l+s+s2}{\PYZdq{}}\PY{p}{,} \PY{l+s+s2}{\PYZdq{}}\PY{l+s+s2}{normalised\PYZus{}temperature}\PY{l+s+s2}{\PYZdq{}}\PY{p}{,} \PY{n}{trendline}\PY{o}{=}\PY{n}{model}\PY{o}{.}\PY{n}{predict}\PY{p}{)}    
\end{Verbatim}
\end{tcolorbox}

    \begin{Verbatim}[commandchars=\\\{\}]
Training beginning{\ldots}
Iteration: 0
Iteration: 400
Iteration: 800
Iteration: 1200
Iteration: 1600
Iteration: 2000
Iteration: 2400
Iteration: 2800
Iteration: 3200
Iteration: 3600
Iteration: 4000
Training complete!
Model parameters after training:        -0.00648853,    0.01193327
    \end{Verbatim}

    
    
    Notice how now the model is trained. It's giving more sensible
predictions about January temperatures.

Interestingly, the model shows temperatures going up over time. Perhaps
we need to stop feeding grain to our elk earlier in the year!

\hypertarget{summary}{%
\subsection{Summary}\label{summary}}

In this exercise, we split up supervised learning into its individual
stages to see what's going on in code when we use third-party libraries.
The important point to take away is how these pieces fit together. Note
that most parts of this process require data.


    % Add a bibliography block to the postdoc
    
    
    
\end{document}
